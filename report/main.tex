% LaTeX template for reports
% Author: Adam Jaamour
% Last updated: 13/04/2020

% ------------------- IMPORTS -------------------
\documentclass[letterpaper,12pt]{article}
\usepackage{tabularx} % extra features for tabular environment
\usepackage{amsmath}  % improve maths presentation
\usepackage{amssymb} % maths symbols
\usepackage{graphicx} % takes care of graphic including machinery
\usepackage[margin=0.95in,letterpaper]{geometry} % decreases margins
\usepackage{cite} % takes care of citations
\usepackage[titletoc,title]{appendix} % takes care of appendices
\usepackage{listings} % code representation
\usepackage{pdflscape}
\usepackage{csquotes} % for quoting existing work
\usepackage{color} % defines colours for code listings
\usepackage{comment} % allows for block of comments
\usepackage{gensymb} % degree symbol
\usepackage[table,xcdraw]{xcolor} % table colouring
\usepackage[cc]{titlepic}  % allows a pic to be included in the title page
\usepackage[final]{hyperref} % adds hyper links inside the generated pdf file
\usepackage{pdfpages} % include pdfs

% ------------------- CODING STYLE -------------------
\definecolor{codegreen}{rgb}{0,0.6,0}
\definecolor{codegray}{rgb}{0.5,0.5,0.5}
\definecolor{backcolour}{rgb}{0.95,0.95,0.92}
\lstdefinestyle{mystyle}{
    backgroundcolor=\color{backcolour},   
    commentstyle=\color{codegreen},
    keywordstyle=\color{blue},
    numberstyle=\tiny\color{codegray},
    basicstyle=\footnotesize,
    breakatwhitespace=false,         
    breaklines=true,                 
    captionpos=b,                    
    keepspaces=true,                 
    numbersep=5pt,                  
    showspaces=false,                
    showstringspaces=false,
    showtabs=false,                  
    tabsize=4
}
\lstset{style=mystyle}

% ------------------- HEADINGS -------------------

\begin{document}

\title{
    CS5014 Machine Learning\\Practical 2 Report\\
    \begin{large}
    University of St Andrews - School of Computer Science
    \end{large}
}
\titlepic{\includegraphics[width=0.3\linewidth]{figures/st-andrews-logo.jpeg}}
\author{Student ID: 150014151}
\date{1st May, 2020}
\maketitle
\newpage

\tableofcontents
\newpage

% ------------------- INTRODUCTION --------------------

\section{Introduction}
\label{sec:introduction}

Todo\\

Content-Based Video Retrieval for Pattern Matching Video Clips \cite{Jaamour2019}.

% ------------------- PART 2: XXXX --------------------

\section{System Architecture}
\label{sec:system-architecture}

\subsection{Project structure}

todo

\subsection{Execution flow}

todo
% ------------------- PART 3: XXXX --------------------

\section{Methodology \& Design decisions}
\label{sec:methodology-design}

\subsection{Data loading}

The three provided datasets are directly loaded into distinct Pandas DataFrames\footnote{Pandas DataFrames: \url{https://pandas.pydata.org/pandas-docs/stable/reference/api/pandas.DataFrame.html}}  by using the \textit{read\_csv} function. However, this method takes a long time due to the file size (X\_train.csv is 1.5GB large). To speed this up, the DataFrames are serialised and saved using Pickle files, boosting the loading times to X second.

\subsection{Data visualisation \& Analysis}

(note: takes ~78s to run)

\subsubsection{Classes distribution}

unbalanced dataset in binary and multi, which have  a large impact on  the scoring method chosen (more in  future sections).

\subsubsection{Correlation}

3 sets of features in the data: HoG, RGB, ND
HoG and RGB are important, ND is not (no correlation)

\subsubsection{HoG features}

HoG features reconstructed into images

\subsubsection{RGB colour histograms}

for a single image and aggregate of all images in training set

\subsection{Input preparation}

standardise: show how different the  scales are (min  and max?)
PCA

show effects of PCA (elbow graph, times, sizes)

\subsection{Fitting}

\subsubsection{Performance metrics}

todo

\subsubsection{K-fold cross validation}

todo

\subsection{Classification models}

SVC and logistic regression  don't natively work with multilabel classification, whereas SGD does.

\subsection{Model fine-tuning}

todo

\subsubsection{Training  execution flow}

todo

% ------------------- CONCLUSION --------------------

\section{Evaluation \& Critical discussion}
\label{sec:evaluation}

\subsection{Selecting the best classifier}

todo

\subsection{Final test}

todo

\subsection{Critical discussion}


% -------------------- APPENDIX --------------------

\begin{appendices}

\clearpage

\bibliographystyle{plain}
\bibliography{bibliography}

% --------------------

\clearpage
\section{Test}
\label{sec:appendix-demo-figure}

test

% ------------------------

\end{appendices}
\end{document}